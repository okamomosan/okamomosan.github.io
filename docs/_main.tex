% Options for packages loaded elsewhere
\PassOptionsToPackage{unicode}{hyperref}
\PassOptionsToPackage{hyphens}{url}
%
\documentclass[
]{book}
\usepackage{amsmath,amssymb}
\usepackage{lmodern}
\usepackage{iftex}
\ifPDFTeX
  \usepackage[T1]{fontenc}
  \usepackage[utf8]{inputenc}
  \usepackage{textcomp} % provide euro and other symbols
\else % if luatex or xetex
  \usepackage{unicode-math}
  \defaultfontfeatures{Scale=MatchLowercase}
  \defaultfontfeatures[\rmfamily]{Ligatures=TeX,Scale=1}
\fi
% Use upquote if available, for straight quotes in verbatim environments
\IfFileExists{upquote.sty}{\usepackage{upquote}}{}
\IfFileExists{microtype.sty}{% use microtype if available
  \usepackage[]{microtype}
  \UseMicrotypeSet[protrusion]{basicmath} % disable protrusion for tt fonts
}{}
\makeatletter
\@ifundefined{KOMAClassName}{% if non-KOMA class
  \IfFileExists{parskip.sty}{%
    \usepackage{parskip}
  }{% else
    \setlength{\parindent}{0pt}
    \setlength{\parskip}{6pt plus 2pt minus 1pt}}
}{% if KOMA class
  \KOMAoptions{parskip=half}}
\makeatother
\usepackage{xcolor}
\usepackage{color}
\usepackage{fancyvrb}
\newcommand{\VerbBar}{|}
\newcommand{\VERB}{\Verb[commandchars=\\\{\}]}
\DefineVerbatimEnvironment{Highlighting}{Verbatim}{commandchars=\\\{\}}
% Add ',fontsize=\small' for more characters per line
\usepackage{framed}
\definecolor{shadecolor}{RGB}{248,248,248}
\newenvironment{Shaded}{\begin{snugshade}}{\end{snugshade}}
\newcommand{\AlertTok}[1]{\textcolor[rgb]{0.94,0.16,0.16}{#1}}
\newcommand{\AnnotationTok}[1]{\textcolor[rgb]{0.56,0.35,0.01}{\textbf{\textit{#1}}}}
\newcommand{\AttributeTok}[1]{\textcolor[rgb]{0.77,0.63,0.00}{#1}}
\newcommand{\BaseNTok}[1]{\textcolor[rgb]{0.00,0.00,0.81}{#1}}
\newcommand{\BuiltInTok}[1]{#1}
\newcommand{\CharTok}[1]{\textcolor[rgb]{0.31,0.60,0.02}{#1}}
\newcommand{\CommentTok}[1]{\textcolor[rgb]{0.56,0.35,0.01}{\textit{#1}}}
\newcommand{\CommentVarTok}[1]{\textcolor[rgb]{0.56,0.35,0.01}{\textbf{\textit{#1}}}}
\newcommand{\ConstantTok}[1]{\textcolor[rgb]{0.00,0.00,0.00}{#1}}
\newcommand{\ControlFlowTok}[1]{\textcolor[rgb]{0.13,0.29,0.53}{\textbf{#1}}}
\newcommand{\DataTypeTok}[1]{\textcolor[rgb]{0.13,0.29,0.53}{#1}}
\newcommand{\DecValTok}[1]{\textcolor[rgb]{0.00,0.00,0.81}{#1}}
\newcommand{\DocumentationTok}[1]{\textcolor[rgb]{0.56,0.35,0.01}{\textbf{\textit{#1}}}}
\newcommand{\ErrorTok}[1]{\textcolor[rgb]{0.64,0.00,0.00}{\textbf{#1}}}
\newcommand{\ExtensionTok}[1]{#1}
\newcommand{\FloatTok}[1]{\textcolor[rgb]{0.00,0.00,0.81}{#1}}
\newcommand{\FunctionTok}[1]{\textcolor[rgb]{0.00,0.00,0.00}{#1}}
\newcommand{\ImportTok}[1]{#1}
\newcommand{\InformationTok}[1]{\textcolor[rgb]{0.56,0.35,0.01}{\textbf{\textit{#1}}}}
\newcommand{\KeywordTok}[1]{\textcolor[rgb]{0.13,0.29,0.53}{\textbf{#1}}}
\newcommand{\NormalTok}[1]{#1}
\newcommand{\OperatorTok}[1]{\textcolor[rgb]{0.81,0.36,0.00}{\textbf{#1}}}
\newcommand{\OtherTok}[1]{\textcolor[rgb]{0.56,0.35,0.01}{#1}}
\newcommand{\PreprocessorTok}[1]{\textcolor[rgb]{0.56,0.35,0.01}{\textit{#1}}}
\newcommand{\RegionMarkerTok}[1]{#1}
\newcommand{\SpecialCharTok}[1]{\textcolor[rgb]{0.00,0.00,0.00}{#1}}
\newcommand{\SpecialStringTok}[1]{\textcolor[rgb]{0.31,0.60,0.02}{#1}}
\newcommand{\StringTok}[1]{\textcolor[rgb]{0.31,0.60,0.02}{#1}}
\newcommand{\VariableTok}[1]{\textcolor[rgb]{0.00,0.00,0.00}{#1}}
\newcommand{\VerbatimStringTok}[1]{\textcolor[rgb]{0.31,0.60,0.02}{#1}}
\newcommand{\WarningTok}[1]{\textcolor[rgb]{0.56,0.35,0.01}{\textbf{\textit{#1}}}}
\usepackage{longtable,booktabs,array}
\usepackage{calc} % for calculating minipage widths
% Correct order of tables after \paragraph or \subparagraph
\usepackage{etoolbox}
\makeatletter
\patchcmd\longtable{\par}{\if@noskipsec\mbox{}\fi\par}{}{}
\makeatother
% Allow footnotes in longtable head/foot
\IfFileExists{footnotehyper.sty}{\usepackage{footnotehyper}}{\usepackage{footnote}}
\makesavenoteenv{longtable}
\usepackage{graphicx}
\makeatletter
\def\maxwidth{\ifdim\Gin@nat@width>\linewidth\linewidth\else\Gin@nat@width\fi}
\def\maxheight{\ifdim\Gin@nat@height>\textheight\textheight\else\Gin@nat@height\fi}
\makeatother
% Scale images if necessary, so that they will not overflow the page
% margins by default, and it is still possible to overwrite the defaults
% using explicit options in \includegraphics[width, height, ...]{}
\setkeys{Gin}{width=\maxwidth,height=\maxheight,keepaspectratio}
% Set default figure placement to htbp
\makeatletter
\def\fps@figure{htbp}
\makeatother
\setlength{\emergencystretch}{3em} % prevent overfull lines
\providecommand{\tightlist}{%
  \setlength{\itemsep}{0pt}\setlength{\parskip}{0pt}}
\setcounter{secnumdepth}{5}
\usepackage{booktabs}
\ifLuaTeX
  \usepackage{selnolig}  % disable illegal ligatures
\fi
\usepackage[]{natbib}
\bibliographystyle{plainnat}
\IfFileExists{bookmark.sty}{\usepackage{bookmark}}{\usepackage{hyperref}}
\IfFileExists{xurl.sty}{\usepackage{xurl}}{} % add URL line breaks if available
\urlstyle{same} % disable monospaced font for URLs
\hypersetup{
  pdftitle={行動経済学の実証分析},
  pdfauthor={Tomohisa OKADA},
  hidelinks,
  pdfcreator={LaTeX via pandoc}}

\title{行動経済学の実証分析}
\author{Tomohisa OKADA}
\date{2023-04-26}

\usepackage{amsthm}
\newtheorem{theorem}{Theorem}[chapter]
\newtheorem{lemma}{Lemma}[chapter]
\newtheorem{corollary}{Corollary}[chapter]
\newtheorem{proposition}{Proposition}[chapter]
\newtheorem{conjecture}{Conjecture}[chapter]
\theoremstyle{definition}
\newtheorem{definition}{Definition}[chapter]
\theoremstyle{definition}
\newtheorem{example}{Example}[chapter]
\theoremstyle{definition}
\newtheorem{exercise}{Exercise}[chapter]
\theoremstyle{definition}
\newtheorem{hypothesis}{Hypothesis}[chapter]
\theoremstyle{remark}
\newtheorem*{remark}{Remark}
\newtheorem*{solution}{Solution}
\begin{document}
\maketitle

{
\setcounter{tocdepth}{1}
\tableofcontents
}
\hypertarget{ready}{%
\chapter{Ready?}\label{ready}}

\hypertarget{rstudioux3068ux306f}{%
\section{R(Studio)とは?}\label{rstudioux3068ux306f}}

\begin{itemize}
\tightlist
\item
  Rは、オープンソースで利用可能な統計解析に特化したプログラミング言語
\item
  RStudioは、Rを使うための統合開発環境(なくてもRは使えるが、あるとめちゃ便利!)
\end{itemize}

\hypertarget{tips}{%
\section{Tips!}\label{tips}}

\begin{itemize}
\tightlist
\item
  Rに関するお役立ちサイトや書籍をまとめる
\end{itemize}

\begin{enumerate}
\def\labelenumi{\arabic{enumi}.}
\item
  \href{https://tomoecon.github.io/R_for_graduate_thesis/}{「卒業論文のためのR入門」 by 森知春先生}\\
  Rの使い方と心理統計学の基礎が一挙に学べる!
\item
  \href{https://www.jaysong.net/RBook/datahandling1.html}{「私たちのR」 by 宋財泫先生 \& 矢内勇生先生}\\
  辞書的に使える!Tidyverseを使ったモダンなRのプログラムが学べる!
\end{enumerate}

\hypertarget{rstudioux306eux30a4ux30f3ux30b9ux30c8ux30fcux30eb}{%
\section{R(Studio)のインストール}\label{rstudioux306eux30a4ux30f3ux30b9ux30c8ux30fcux30eb}}

\begin{itemize}
\tightlist
\item
  次のサイトにアクセス( \url{https://posit.co/download/rstudio-desktop/} )
\item
  「1:Install R」からRをインストール
\item
  「2:Install RStudio」からRStudioをインストール
\item
  うまくインストールできない場合は、\href{https://www.jaysong.net/RBook/installation.html}{「私たちのR」}に詳しい解説がある
\end{itemize}

\hypertarget{go}{%
\chapter{Go!!}\label{go}}

\begin{itemize}
\tightlist
\item
  とりあえずまずは使ってみよう!
\end{itemize}

\hypertarget{rstudioux3092ux958bux304f}{%
\section{RStudioを開く}\label{rstudioux3092ux958bux304f}}

\begin{itemize}
\tightlist
\item
  RStudioのアイコンから起動
\item
  次のような、4つの枠(ペーン)に分かれた画面が表示される
\end{itemize}

\includegraphics{./img/2_rstudio_4pane.png}

  \textbf{①左上:}メインの作業場でコードを書くところ\\
  \textbf{②右上:}変数やオブジェクトのリストが表示されたり、バージョン管理を行う\\
  \textbf{③左下:}コードの実行結果が表示されるコンソールや、各種コマンドを実行するターミナルなど\\
  \textbf{④右下:}各種ファイルやパッケージの表示、出力した図の表示など\\
  

\hypertarget{ux30b3ux30f3ux30bdux30fcux30ebux306bux76f4ux63a5ux6253ux3061ux8fbcux3080}{%
\section{コンソールに直接打ち込む}\label{ux30b3ux30f3ux30bdux30fcux30ebux306bux76f4ux63a5ux6253ux3061ux8fbcux3080}}

\begin{itemize}
\tightlist
\item
  RStudioの左下のペーンを見る\\
\item
  Consoleのタブが選択されていることを確認
\item
  最下段の\texttt{\textgreater{}}のあとに\texttt{1+1}と打ち込み\textbf{Enter(macはreturn)}を押す
\item
  \texttt{{[}1{]}\ 2}と返ってくる
\item
  \texttt{2}の部分が\texttt{1+1}の計算結果
\end{itemize}

\hypertarget{rux30b9ux30afux30eaux30d7ux30c8ux3092ux4f7fux3046}{%
\section{Rスクリプトを使う}\label{rux30b9ux30afux30eaux30d7ux30c8ux3092ux4f7fux3046}}

\begin{itemize}
\tightlist
\item
  コンソールに書いたコードは、Rstudioを終了すると消える\\
  (実際に終了して再起動してみよう)\\
\item
  だから、保存可能な\textbf{Rスクリプト}にコードを書くことが多い
\end{itemize}

\hypertarget{rux30b9ux30afux30eaux30d7ux30c8ux306bux66f8ux304f}{%
\paragraph*{Rスクリプトに書く:}\label{rux30b9ux30afux30eaux30d7ux30c8ux306bux66f8ux304f}}
\addcontentsline{toc}{paragraph}{Rスクリプトに書く:}

\begin{itemize}
\tightlist
\item
  RStudioの左上の
  をクリックし、「R Script」を選択
\item
  空のRスクリプトファイルが作成され、左上のペーンに表示される
\item
  1行目に\texttt{1+1}と入力して\textbf{ctrl+Enter(macはcommand+return)}を押す\\
  ( をクリックしても良い)
\item
  左下のペーンのコンソールに結果(\texttt{{[}1{]}\ 2})が表示される
\end{itemize}

\hypertarget{ux4fddux5b58}{%
\paragraph*{保存:}\label{ux4fddux5b58}}
\addcontentsline{toc}{paragraph}{保存:}

\begin{itemize}
\tightlist
\item
  Rスクリプトは\textbf{ctrl+s(macはcommand+s)}で好きな時に保存できる
\item
  初回はファイル名と保存場所も決める
\item
  試しに、保存したRスクリプトのタブを閉じてみよう
  (タブの右側の×をクリック)
\item
  左上の をクリックし、保存したRスクリプトのファイルを選択\\
\item
  Rスクリプトが先ほど保存した状態で開く
\end{itemize}

\hypertarget{ux3055ux3089ux306bux66f8ux304f}{%
\paragraph*{さらに書く!:}\label{ux3055ux3089ux306bux66f8ux304f}}
\addcontentsline{toc}{paragraph}{さらに書く!:}

\begin{itemize}
\tightlist
\item
  2行目に\texttt{5-2}と入力する
\item
  2行目にカーソルを合わせて\textbf{ctrl+Enter}を押す
\item
  2行目の計算結果(\texttt{{[}1{]}\ 3})が表示される
\item
  1行目にカーソルを合わせて\textbf{ctrl+Enter}を押す
\item
  1行目の計算結果(\texttt{{[}1{]}\ 2})が表示される
\item
  つまり、実行したい行にカーソルを合わせて\textbf{ctrl+Enter}を押せば良い
\item
  全ての行を一括で実行したいなら\textbf{ctrl+shift+Enter(macはcommand+shift+return)}
\item
  一部分だけ実行したいなら、下図のように実行したい行だけ選択して\textbf{ctrl+Enter}
\end{itemize}

\hypertarget{rux30b9ux30afux30eaux30d7ux30c8ux3092ux4f7fux3046ux30e1ux30eaux30c3ux30c8}{%
\paragraph*{Rスクリプトを使うメリット:}\label{rux30b9ux30afux30eaux30d7ux30c8ux3092ux4f7fux3046ux30e1ux30eaux30c3ux30c8}}
\addcontentsline{toc}{paragraph}{Rスクリプトを使うメリット:}

\begin{enumerate}
\def\labelenumi{\arabic{enumi}.}
\tightlist
\item
  コードが保存できる(毎回書き直さなくて良い!)
\item
  長く複雑なコードを書いたり、管理するのが楽\\
\item
  他の人に配布できる(分析を再現してもらいやすい)\\
  などなど\ldots{}
\end{enumerate}

\hypertarget{ux30d7ux30edux30b8ux30a7ux30afux30c8ux7ba1ux7406}{%
\section{プロジェクト管理}\label{ux30d7ux30edux30b8ux30a7ux30afux30c8ux7ba1ux7406}}

\begin{itemize}
\tightlist
\item
  研究プロジェクトが進むと、ひとつのRスクリプトだけでは管理しきれなくなる
\item
  \textbf{プロジェクト}を使えば、複数のRスクリプトや関連データなどを一つのフォルダにまとめて効率よく管理できる
\end{itemize}

\hypertarget{ux30d7ux30edux30b8ux30a7ux30afux30c8ux306eux4f5cux6210}{%
\paragraph*{プロジェクトの作成:}\label{ux30d7ux30edux30b8ux30a7ux30afux30c8ux306eux4f5cux6210}}
\addcontentsline{toc}{paragraph}{プロジェクトの作成:}

\begin{itemize}
\tightlist
\item
  左上の をクリックし、「New Directry」→「New Project」の順に選択\\
\item
  次の画面で①プロジェクト名と、その②作成場所を指定して③「Create Project」
\end{itemize}

\begin{itemize}
\tightlist
\item
  指定した場所に、指定したプロジェクト名のフォルダができているのを確認しよう
\item
  そのフォルダの中に「プロジェクト名.Rproj」というファイルができているのを確認しよう
\item
  以下は、macでDocumentフォルダ内にsugoi\_projectというプロジェクトを作った例
\end{itemize}

\begin{itemize}
\tightlist
\item
  初めはご利益がわかりにくいが、研究プロジェクトごとにプロジェクトを作るクセをつけよう
\item
  そして、次の心得に従い、プロジェクト上で作業をするようにしよう!
\end{itemize}

\begin{enumerate}
\def\labelenumi{\arabic{enumi}.}
\tightlist
\item
  毎回「プロジェクト名.Rproj」をダブルクリックしてRstudioを起動
\item
  関連するファイルはプロジェクト名のフォルダにまとめて一元管理
\end{enumerate}

\hypertarget{cross}{%
\chapter{Cross-references}\label{cross}}

Cross-references make it easier for your readers to find and link to elements in your book.

\hypertarget{chapters-and-sub-chapters}{%
\section{Chapters and sub-chapters}\label{chapters-and-sub-chapters}}

There are two steps to cross-reference any heading:

\begin{enumerate}
\def\labelenumi{\arabic{enumi}.}
\tightlist
\item
  Label the heading: \texttt{\#\ Hello\ world\ \{\#nice-label\}}.

  \begin{itemize}
  \tightlist
  \item
    Leave the label off if you like the automated heading generated based on your heading title: for example, \texttt{\#\ Hello\ world} = \texttt{\#\ Hello\ world\ \{\#hello-world\}}.
  \item
    To label an un-numbered heading, use: \texttt{\#\ Hello\ world\ \{-\#nice-label\}} or \texttt{\{\#\ Hello\ world\ .unnumbered\}}.
  \end{itemize}
\item
  Next, reference the labeled heading anywhere in the text using \texttt{\textbackslash{}@ref(nice-label)}; for example, please see Chapter \ref{cross}.

  \begin{itemize}
  \tightlist
  \item
    If you prefer text as the link instead of a numbered reference use: \protect\hyperlink{cross}{any text you want can go here}.
  \end{itemize}
\end{enumerate}

\hypertarget{captioned-figures-and-tables}{%
\section{Captioned figures and tables}\label{captioned-figures-and-tables}}

Figures and tables \emph{with captions} can also be cross-referenced from elsewhere in your book using \texttt{\textbackslash{}@ref(fig:chunk-label)} and \texttt{\textbackslash{}@ref(tab:chunk-label)}, respectively.

See Figure \ref{fig:nice-fig}.

\begin{Shaded}
\begin{Highlighting}[]
\FunctionTok{par}\NormalTok{(}\AttributeTok{mar =} \FunctionTok{c}\NormalTok{(}\DecValTok{4}\NormalTok{, }\DecValTok{4}\NormalTok{, .}\DecValTok{1}\NormalTok{, .}\DecValTok{1}\NormalTok{))}
\FunctionTok{plot}\NormalTok{(pressure, }\AttributeTok{type =} \StringTok{\textquotesingle{}b\textquotesingle{}}\NormalTok{, }\AttributeTok{pch =} \DecValTok{19}\NormalTok{)}
\end{Highlighting}
\end{Shaded}

\begin{figure}

{\centering \includegraphics[width=0.8\linewidth]{_main_files/figure-latex/nice-fig-1} 

}

\caption{Here is a nice figure!}\label{fig:nice-fig}
\end{figure}

Don't miss Table \ref{tab:nice-tab}.

\begin{Shaded}
\begin{Highlighting}[]
\NormalTok{knitr}\SpecialCharTok{::}\FunctionTok{kable}\NormalTok{(}
  \FunctionTok{head}\NormalTok{(pressure, }\DecValTok{10}\NormalTok{), }\AttributeTok{caption =} \StringTok{\textquotesingle{}Here is a nice table!\textquotesingle{}}\NormalTok{,}
  \AttributeTok{booktabs =} \ConstantTok{TRUE}
\NormalTok{)}
\end{Highlighting}
\end{Shaded}

\begin{table}

\caption{\label{tab:nice-tab}Here is a nice table!}
\centering
\begin{tabular}[t]{rr}
\toprule
temperature & pressure\\
\midrule
0 & 0.0002\\
20 & 0.0012\\
40 & 0.0060\\
60 & 0.0300\\
80 & 0.0900\\
\addlinespace
100 & 0.2700\\
120 & 0.7500\\
140 & 1.8500\\
160 & 4.2000\\
180 & 8.8000\\
\bottomrule
\end{tabular}
\end{table}

\hypertarget{parts}{%
\chapter{Parts}\label{parts}}

You can add parts to organize one or more book chapters together. Parts can be inserted at the top of an .Rmd file, before the first-level chapter heading in that same file.

Add a numbered part: \texttt{\#\ (PART)\ Act\ one\ \{-\}} (followed by \texttt{\#\ A\ chapter})

Add an unnumbered part: \texttt{\#\ (PART\textbackslash{}*)\ Act\ one\ \{-\}} (followed by \texttt{\#\ A\ chapter})

Add an appendix as a special kind of un-numbered part: \texttt{\#\ (APPENDIX)\ Other\ stuff\ \{-\}} (followed by \texttt{\#\ A\ chapter}). Chapters in an appendix are prepended with letters instead of numbers.

\hypertarget{footnotes-and-citations}{%
\chapter{Footnotes and citations}\label{footnotes-and-citations}}

\hypertarget{footnotes}{%
\section{Footnotes}\label{footnotes}}

Footnotes are put inside the square brackets after a caret \texttt{\^{}{[}{]}}. Like this one \footnote{This is a footnote.}.

\hypertarget{citations}{%
\section{Citations}\label{citations}}

Reference items in your bibliography file(s) using \texttt{@key}.

For example, we are using the \textbf{bookdown} package \citep{R-bookdown} (check out the last code chunk in index.Rmd to see how this citation key was added) in this sample book, which was built on top of R Markdown and \textbf{knitr} \citep{xie2015} (this citation was added manually in an external file book.bib).
Note that the \texttt{.bib} files need to be listed in the index.Rmd with the YAML \texttt{bibliography} key.

The RStudio Visual Markdown Editor can also make it easier to insert citations: \url{https://rstudio.github.io/visual-markdown-editing/\#/citations}

\hypertarget{blocks}{%
\chapter{Blocks}\label{blocks}}

\hypertarget{equations}{%
\section{Equations}\label{equations}}

Here is an equation.

\begin{equation} 
  f\left(k\right) = \binom{n}{k} p^k\left(1-p\right)^{n-k}
  \label{eq:binom}
\end{equation}

You may refer to using \texttt{\textbackslash{}@ref(eq:binom)}, like see Equation \eqref{eq:binom}.

\hypertarget{theorems-and-proofs}{%
\section{Theorems and proofs}\label{theorems-and-proofs}}

Labeled theorems can be referenced in text using \texttt{\textbackslash{}@ref(thm:tri)}, for example, check out this smart theorem \ref{thm:tri}.

\begin{theorem}
\protect\hypertarget{thm:tri}{}\label{thm:tri}For a right triangle, if \(c\) denotes the \emph{length} of the hypotenuse
and \(a\) and \(b\) denote the lengths of the \textbf{other} two sides, we have
\[a^2 + b^2 = c^2\]
\end{theorem}

Read more here \url{https://bookdown.org/yihui/bookdown/markdown-extensions-by-bookdown.html}.

\hypertarget{callout-blocks}{%
\section{Callout blocks}\label{callout-blocks}}

The R Markdown Cookbook provides more help on how to use custom blocks to design your own callouts: \url{https://bookdown.org/yihui/rmarkdown-cookbook/custom-blocks.html}

\hypertarget{sharing-your-book}{%
\chapter{Sharing your book}\label{sharing-your-book}}

\hypertarget{publishing}{%
\section{Publishing}\label{publishing}}

HTML books can be published online, see: \url{https://bookdown.org/yihui/bookdown/publishing.html}

\hypertarget{pages}{%
\section{404 pages}\label{pages}}

By default, users will be directed to a 404 page if they try to access a webpage that cannot be found. If you'd like to customize your 404 page instead of using the default, you may add either a \texttt{\_404.Rmd} or \texttt{\_404.md} file to your project root and use code and/or Markdown syntax.

\hypertarget{metadata-for-sharing}{%
\section{Metadata for sharing}\label{metadata-for-sharing}}

Bookdown HTML books will provide HTML metadata for social sharing on platforms like Twitter, Facebook, and LinkedIn, using information you provide in the \texttt{index.Rmd} YAML. To setup, set the \texttt{url} for your book and the path to your \texttt{cover-image} file. Your book's \texttt{title} and \texttt{description} are also used.

This \texttt{gitbook} uses the same social sharing data across all chapters in your book- all links shared will look the same.

Specify your book's source repository on GitHub using the \texttt{edit} key under the configuration options in the \texttt{\_output.yml} file, which allows users to suggest an edit by linking to a chapter's source file.

Read more about the features of this output format here:

\url{https://pkgs.rstudio.com/bookdown/reference/gitbook.html}

Or use:

\begin{Shaded}
\begin{Highlighting}[]
\NormalTok{?bookdown}\SpecialCharTok{::}\NormalTok{gitbook}
\end{Highlighting}
\end{Shaded}


  \bibliography{book.bib,packages.bib}

\end{document}
